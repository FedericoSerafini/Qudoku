\documentclass{article}

% Packages.

% Writing.
\usepackage[utf8]{inputenc}
\usepackage[english]{babel}
\usepackage{enumitem} % Enumerate using letters (\begin{description}[\alph]).

\usepackage{subfiles}

% Drawing, tables and images.
\usepackage{graphicx}
\usepackage{subfigure}
\usepackage{float} % Help in table and image positioning.
\usepackage{qtree}
\usepackage{tikz}
\usepackage{svg}

% Code.
\usepackage{minted}
\usemintedstyle{tango}


% Math.
\usepackage{mathtools} % Overbrackets.
\usepackage{amsmath} % Symbols.
\usepackage{amsthm} % Theorems.
\usepackage{amssymb} % Mathbb{}.
\usepackage{stmaryrd} % Brackets.
\usepackage{semantic}
\usepackage{relsize}
\usepackage{braket}

% Exercises.
\usepackage{exercise, chngcntr}
\counterwithin{Exercise}{section}
\counterwithin{Answer}{section}

% New commands.

% Theorems.
\newtheorem{Osservazione}{Osservazione}[section]
\newtheorem{Definition}{Definition}[section]
\newtheorem{Teorema}{Teorema}[section]
\newtheorem{Lemma}{Lemma}[section]
\newtheorem{Corollario}{Corollario}[section]
\newtheorem{thm}{Thm}[section]

\newcommand*{\uepsilon}{\underline{\epsilon}}
\newcommand*{\uempty}{\underline{\varnothing}}
\newcommand*{\ucdot}{\mathbin{\underline{\mathord{\cdot}}}}
\newcommand*{\lpar}{\underline{(}}
\newcommand*{\rpar}{\underline{)}}
\newcommand*{\ersem}[1]{\llbracket #1 \rrbracket}
\newcommand*{\bigersem}[1]{\bigl\llbracket #1 \bigr\rrbracket}
\newcommand*{\match}{\mathrel{\lessdot}}
\newcommand*{\nmatch}{\mathrel{\not\!\!\lessdot}}
\newcommand*{\Nset}{\mathbb{N}}
\newcommand*{\Rset}{\mathbb{R}}
\newcommand*{\defeq}{\mathrel{:=}}
\newcommand*{\sseq}{\subseteq}
\newcommand*{\ttt}{\texttt}
\newcommand*{\card}[1]{\lvert #1 \rvert}
\newcommand*{\powset}[1]{\wp(#1)}
\newcommand*{\powsetf}[1]{\wp_{f}(#1)}
\newcommand*{\st}{\mathrel{.}} % Such that, to be used with \exists.
\newcommand*{\itc}{\mathrel{:}} % It is the case that, to be used with \forall.
\newcommand*{\hd}{\hat{\delta}}
\newcommand*{\cd}{\check \delta}
\newcommand*{\ep}{\epsilon}
\newcommand*{\epstep}{\epsilon\text{-step}}
\newcommand*{\epclos}{\epsilon\text{-closure}}
\newcommand*{\rel}{\mathrel R}
\newcommand*{\nrel}{\mathrel{\not\!\!R}}
\newcommand*{\qt}{q_{\varnothing}}
\newcommand*{\ra}{\rightarrow}
\newcommand*{\grder}{\stackrel{G}{\Rightarrow}}
\newcommand*{\grderp}{\stackrel{G'}{\Rightarrow}}
\newcommand*{\grders}{\stackrel{G''}{\Rightarrow}}
\newcommand*{\grdef}{\mathrel{\mathord{:}\mathord{:}\mathord{=}}}
\newcommand*{\gralt}{\mathrel{\mid}}
\newcommand*{\defpar}{\rightarrowtail}
\newcommand*{\compl}[1]{\overline{#1}}
\newcommand*{\qmarrow}{\stackrel{\text{\larger{\textbf{?}}}}{\Longrightarrow}}
\newcommand*{\twodots}{\mathrel{. \, .}}


\def\checkmark{\tikz\fill[scale=0.4](0,.35) -- (.25,0) -- (1,.7) -- (.25,.15)
  -- cycle;}

\title
{
  Qudoku: solving Sudoku using Grover's algorithm \\
  \vspace*{1em}
  \large Quantum Computing\\
         Computer Science Master Degree\\
         University of Parma
}

\author{
  Federico Serafini\\
  \normalsize \texttt{federico.serafini@studenti.unipr.it}
}

\date{08/02/2023}

\begin{document}

\begin{titlepage}
  \clearpage\maketitle
  \thispagestyle{empty}
\end{titlepage}


\tableofcontents
\newpage

\section{Introduction}
Many interesting problems that computer science tries to solve are \emph{search
problems} over \emph{structured} and \emph{unstructured} databases.
The best algorithm for searching a winning element $\omega$ in a structured
database, a sorted list of $N$ elements for example, is the \emph{binary
search}: it allows to find $\omega$ in $\mathcal{O} (\log N)$ tries exploiting
data ordering.
Things get more difficult when the search is done over a list of $N$
randomly-placed elements.
Using a classical algorithm there is no way of taking advantage of
the data structure to speed up the search:
in the worst case, where the searched element is at the end of the list, a scan
over all the $N$ elements of the list is required.
In the average case, the number of tries to find $\omega$ is $N/2$.
So, the overall time complexity is $\mathcal{O}(N)$ and this means that time
cost for classical unstructured search algorithms grows linearly with the size
of the input.

In this report, we are going to discuss a new search
algorithm based on quantum computation: Grover's algorithm.
When searching any database (structured or unstructured), Grover's algorithm
time complexity is $\mathcal{O}(\sqrt N)$, which for unstructured
search is a quadratic improvement over the best classical algorithm.
It can serve as a general trick or subroutine to
obtain quadratic run time improvements for a variety of other algorithms
through \emph{amplitude amplification trick}.
In particular, we are going to see a simple example of conversion from a
classical search problem as solving a binary Sudoku, into oracles for Grover's
algorithm.
\begin{figure}[H]
  \centering
  \includesvg[width=200pt]{Img/rg-vs-grover.svg}
  \caption{Time complexity.}
\end{figure}

\section{Basic concepts}

\subsection{Definitions}

\paragraph{Decision problem}
In computability theory and computational complexity theory, a
\emph{decision  problem} is a computational problem that can be posed as a
yes/no question of the input values.
It is traditional to define the decision problem as the set of possible inputs
together with the set of inputs for which the answer is yes.
In our case, it will be useful to formulate the Sudoku as a decision problem,
where the input values are all the $n^{n^2}$ ways of filling the board and the
output is the set of valid solutions.

\paragraph{Oracle}
Many quantum algorithms are based around the analysis of some \emph{oracle}
function $f$: a ``black box" which we can give an input $x$
and receive the corresponding output $f(x)$.
Our objective is to determine some properties of the oracle function using the
minimum number of queries.

\subsection{Phase kickback}
\subsubsection{Rotations and eigenvalues}
Given quantum gate $U$, and it's eigenstate $\ket{x}$, then:
\[
  U \ket{x} = \lambda \ket{x} = e^{2 \pi i \theta} \ket{x},
\]
meaning that if a gate rotates (and only rotates) all the amplitudes of a state
vector by the same amount, then that state is an eigenstate of that gate and
$U$ acting on $\ket{x}$ will add a global phase $\theta$.
For example, performing an $X$-gate on a $\ket{-}$ qubit gives it the phase
$-1$:
\begin{align*}
  X \ket{-}
  & = \tfrac{1}{\sqrt{2}}  \bigl( \ket{0}\bra{1} + \ket{1}\bra{0} \bigr)
      \bigl( \ket{0} - \ket{1} \bigr) \\
  & = \tfrac{1}{\sqrt{2}} \bigl( \ket{0}\bra{1} + \ket{1}\bra{0} \bigr)
      \bigl( \ket{0} - \ket{1} \bigr) \\
  & = \tfrac{1}{\sqrt{2}} \bigl( \ket{0}\braket{1|0} - \ket{0}\braket{1|1}
     + \ket{1}\braket{0|0} - \ket{1}\braket{0|1}  \bigr) \\
  & = \tfrac{1}{\sqrt{2}} \bigl( -\ket{0}  + \ket{1} \bigr) \\
  & =  - \tfrac{1}{\sqrt{2}} \bigl( \ket{0}  - \ket{1} \bigr) = - \ket{-}.
\end{align*}

\subsubsection{Kick back}
\emph{Phase kickback} is where the eigenvalue added by a gate to a qubit is
``kicked back" into a different qubit via a controlled operation.
When the control qubit is in either $\ket{0}$ or $\ket{1}$, this phase affects
the whole state, however it is a global phase and has no observable effects:
\begin{align*}
  CX \ket{-0}
  & = \ket{-} \otimes \ket{0} \\
  & = \ket{-0},
\end{align*}
\begin{align*}
  CX \ket{-1}
  & = X\ket{-} \otimes \ket{1} \\
  & = -\ket{-1}.
\end{align*}
Things get interesting when the control qubit is in a superposition of
$\ket{0}$ and $\ket{1}$.
The component of the control qubit that lies in the direction of  $\ket{1}$
applies this phase factor to the corresponding target qubit. This applied phase
factor in turn introduces a relative phase into the control qubit:
\[
CU \ket{x} \bigl(\alpha \ket{0} + \beta \ket{1} \bigr) =
\ket{x} \bigl(\alpha \ket{0} + \beta e^{2 i \pi \theta} \ket{1} \bigr).
\]
For example:
\begin{align*}
  CX \ket{-+}
  & = \tfrac{1}{\sqrt{2}} \bigl( X\ket{-0} + X\ket{-1} \bigr) \\
  & = \tfrac{1}{\sqrt{2}} \bigl( \ket{-0} + X\ket{-1} \bigr) \\
  & = \tfrac{1}{\sqrt{2}} \bigl( \ket{-0} - \ket{-1} \bigr) \\
  & = \ket{-} \otimes \tfrac{1}{\sqrt{2}} \bigl( \ket{0} - \ket{1} \bigr) =
      \ket{--}.\\
\end{align*}
This interesting quantum effect is a building block in many
famous quantum algorithms, including Grover's search algorithm.

\subsubsection{Oracles}

\paragraph{Classical oracle}
On classical algorithms, oracles can be seen as general functions:
\[
  f: \{\, 0, 1 \,\}^n \mapsto \{\, 0, 1 \,\}^m,
\]
where $n, m \in \Nset$ are the input and output length.

\paragraph{Boolean oracle}
In quantum computation, one of the main forms that oracles take is that of
\emph{boolean oracles}, described by the following unitary evolution:
\[
  U_f \bigl( \ket{y} \otimes \ket{x} \bigr) =  \ket{f(x) \oplus x},
\]
where:
\begin{itemize}
  \item
  $x$ is the input register ($n$ qubits, $n \geq 1$);
  \item
  $y$ is the output register ($m$ qubits, $m \geq 1$);
  \item
  $\oplus$ is an exclusive or.
\end{itemize}
Note that the result of applying $U_f$ depends on:
\begin{enumerate}
  \item
  $U_f$ definition;
  \item
  initial contents of both input and output registers, in particular,
  if $\ket{y} = \ket{0^m}$ then $\ket{f(x) \oplus y} = \ket{f(x)}$.
\end{enumerate}
\begin{figure}[H]
  \centering
  \includesvg{Img/grover-boolean-oracle.svg}
  \caption{Boolean oracle.}
\end{figure}

\paragraph{Phase oracle}
Another form of oracle used in quantum computation is the \emph{phase oracle},
defined as:
\[
  P_f \ket{x} = (-1)^{f(x)} \ket{x}.
\]
A phase oracle can be realized using boolean oracle and the phase kickback
mechanism:
\[
  U_f \bigl( \ket{y} \otimes \ket{x} \bigr)
  = U_f \bigl( \ket{-} \otimes \ket{x} \bigr)
  = \bigl( I \otimes  P_f \bigr) \bigl( \ket{-} \otimes \ket{x} \bigr)
  = \ket{-} \otimes P_f \ket{x}
  = P_f \ket{x}.
\]
Input $x$ controls an $X$ rotation targeting the output qubit $y$ in state
$\ket{-}$:
an $X$ rotation is applied to $\ket{-}$ if $f(x) = 1$, resulting in a kickback
of the phase $-1$ on the state $\ket{x}$.
Output qubit $y$ (whose state is left unchanged by the whole
process and can safely be ignored) it is called \emph{ancilla} qubit.
\begin{figure}[H]
  \centering
  \includesvg{Img/grover-phase-oracle.svg}
  \caption{Phase oracle.}
\end{figure}

\section{Grover's algorithm}

Now we are going to see how the Grover's algorithm works when searching for a
winning item $\omega$ among $N$ items.
Before diving into the details it can
be useful to visualize the shape of the circuit that implements it:
\begin{figure}[H]
  \centering
  \includegraphics[width=345pt]{Img/grover-circuit-high-level.png}
  \caption{Grover's circuit.}
\end{figure}

\subsection{Preparation}
Before looking at the list of items, we have no idea where the winning item
is. Therefore, any guess of its location is as good as any other, which can be
expressed in a equal superposition of
  every possible input:
  \[
    H^{\otimes n} \ket{x} = \ket{s} = \frac{1}{\sqrt{2^n}} \sum_{k \in
    \{0, 1\}^n}
    \ket{k}
  \]
  We can create this superposition by applying a H-gate to each input qubit.

\subsection{Grover oracle}
Grover's oracle $U_\omega$ it is a phase oracle, so its application on an input
state $\ket{x}$ can be described as:
\[
  U_f \ket{x} = (-1)^{f(x)} \ket{x}.
\]
First thing we need to define is the \emph{indicator function} $f$.
A strong point of this algorithm is that it is \emph{general}:
definition of $f$ can vary depending on the search problem we are addressing.
In our case, the indicator function $f$ is defined as:
\[
  f \ket{x} = \bigg\{
  \begin{aligned}
    1, && \text{if} \; x = \omega; \\
    0, && \text{otherwise}.
  \end{aligned}
\]
In general, there are many computational problems in which it’s difficult to
find a solution, but relatively easy to verify a solution; let $W$ be the
set of all valid solutions of a problem, then $f$ will be the function that
tell us if a candidate solution $x$ it is a valid solution or not.
\[
  f \ket{x} = \bigg\{
  \begin{aligned}
    1, && \text{if} \; x \in W; \\
    0, && \text{otherwise}.
  \end{aligned}
\]
Given $f$, we can observe that Grove's oracle $U_\omega$ adds a negative
phase only to the winning element:
\[
  U_\omega \ket{x} = \bigg\{
  \begin{aligned}
    - \ket{x}, && \text{if} \; x = \omega; \\
    \ket{x},   && \text{otherwise}.
  \end{aligned}
\]
Another way of representing $U_\omega$ is taking an identity matrix $I$ and
adding a $-1$ phase to the winning element:
\[
  U_\omega =
  \begin{bmatrix}
    (-1)^{f(0)} &   0         & \cdots &   0         \\
    0           & (-1)^{f(1)} & \cdots &   0         \\
    \vdots      &   0         & \ddots & \vdots      \\
    0           &   0         & \cdots & (-1)^{f(2^n-1)} \\
  \end{bmatrix}
\]

\subsection{Amplitude amplification}
Finally it is time to apply the \emph{amplitude amplification} trick.
A procedure that amplifies the amplitude of the winning element,
which shrinks the other items' amplitude, so that measuring the final state
will return the right item with high probability.
There a nice geometrical interpretation of the amplitude amplification in terms
of two reflections, which generate a rotation in a two-dimensional plane.

\paragraph{Step 1}
Consider the vectors corresponding to the winner $\ket{\omega}$ and the
uniform superposition $\ket{s}$.
These two vectors span a two-dimensional plane.
They are not perpendicular because $\ket{\omega}$
occurs in the superposition $\ket{s}$ with amplitude $\frac{1}{\sqrt{2^n}}$
as well. We can, however, introduce an additional state $\ket{s'}$,
that is in the span of these two vectors and is perpendicular to
$\ket{\omega}$ and (obtained from $\ket{s}$ by removing $\ket{\omega}$):
\[
  \ket{s'} = \tfrac{1}{\sqrt{2^{n-1}}} \sum_{k \neq \omega} \ket{k}.
\]
Then $\ket{s}$ is a linear combination of $\ket{\omega}$ and $\ket{s'}$:
\[
  \ket{s} = \tfrac{1}{\sqrt{2^n}}  \ket{\omega} + \sqrt{\tfrac{2^n-1}{2^n}}
  \ket{s'} = \sin{\theta} \ket{\omega} + \cos{\theta} \ket{s'},
\]
where $\theta = \arcsin{\braket{s|\omega}} = \tfrac{1}{\sqrt{2^n}}
       = \tfrac{1}{\sqrt N}$ is the amplitude af all the different states.
\begin{figure}[H]
  \centering
  \includegraphics[width=345pt]{Img/grover-step1.jpg}
  \caption{$\ket{s} = H^{{\otimes}^n} \ket{x}$.}
\end{figure}

\paragraph{Step 2}
Now we apply the oracle $U_\omega$ to the superposition state
$\ket{s}$.
Geometrically, this corresponds to a reflection of the state $\ket{s}$
about $\ket{s'}$: the amplitude in front of the $\ket{\omega}$ becomes
negative, which in turn means that the average amplitude has been lowered.
\begin{figure}[H]
  \centering
  \includegraphics[width=345pt]{Img/grover-step2.jpg}
  \caption{$U_\omega \ket{s}$.}
\end{figure}

\paragraph{Step 3}
Apply the \emph{diffuser} $U_s = 2\ket{s}\bra{s} -I$ to the state $\ket{s}$.
Since this is a reflection about the state $\ket{s}$, we want to add a negative
phase to every state orthogonal to $\ket{s}$.
One way to do this is:
\begin{enumerate}
  \item
  transform $\ket{s} \rightarrow \ket{0^{\otimes n}}$ applying Hadamard gates;
  \item
  apply a negative phase to the states orthogonal to $\ket{0^{\otimes n}}$
  using $U_0 =
  X^{\otimes n} (MCZ) X^{\otimes n}$;
  \item
  transform $\ket{0^{\otimes n}} \rightarrow \ket{s}$ applying H-gate again.
\end{enumerate}
Putting all together $U_s = H^{\otimes n} X^{\otimes n} (MCZ) X^{\otimes n}
H^{\otimes n} = H^{\otimes n} U_0 H^{\otimes n}$.
\begin{figure}[H]
  \centering
  \includegraphics[width=345pt]{Img/grover-step3.jpg}
  \caption{$U_s U_\omega \ket{s}$}
\end{figure}
The transformation $U_s U_\omega \ket{s}$ rotates $\ket{s}$
closer towards the winner $\ket{\omega}$.
The action of the reflection
in the amplitude bar diagram can be understood as a reflection about the
average amplitude. Since the average amplitude has been lowered by the first
reflection, this transformation boosts the negative amplitude of
$\ket{\omega}$ to roughly three times its original value, while it decreases
the other amplitudes. We then go to step 2 to repeat the application.

\subsection{Repetitions}
\subsubsection{Single element}
Each application of steps 2 and 3, rotates $\ket{s}$
closer towards the winner $\ket{\omega}$ of an angle $2 \theta$.
We would like to find the minimum number of repetitions that brings us as close
as possible to $\ket{w}$: a $90^\circ$ rotation taking into account the initial
position of $\ket{S}$.
\begin{align*}
  & t(2 \theta) = \frac{\pi}{2} - \theta, \\
  & t = \frac{\pi}{4 \theta} - \frac{\theta}{2\theta}, \\
  & t = \frac{\pi}{4} \sqrt{N}- \frac{1}{2} = \mathcal O(\sqrt{N}).
\end{align*}

\subsubsection{Multiple element}
Let $M$ be the number of winning (or marked) elements.
Then the winning state $\ket{\omega}$ is defined as:
\[
  \ket{\omega} = \frac{1}{\sqrt{M}} \sum_{i = 1}^{M} \ket{\omega_i}.
\]
Then:
\begin{itemize}
  \item
  $
    \ket{s'} =
    \frac{1}{\sqrt{N-M}} \sum_{k \notin \{ \omega_1 \ldots \omega_M \}} \ket{k};
  $
  \item
  $
    \ket{s} =
    \sqrt{\frac{M}{N}} \ket{\omega} + \sqrt{\frac{N-M}{N}} \ket{s'}
    = \sin\theta \ket{\omega} + \cos{\theta} \ket{s'}.
  $
\end{itemize}
Angle $\theta$ becomes larger:
\[
  \theta = \arcsin \braket{s|w} = \sqrt{\frac{M}{N}},
\]
and $t$ reduces:
\begin{align*}
  & t(2 \theta) = \frac{\pi}{2} - \theta, \\
  & t = \frac{\pi}{4 \theta} - \frac{\theta}{2\theta}, \\
  & t = \frac{\pi}{4} \sqrt{\frac{N}{M}}- \frac{1}{2}
    = \mathcal O\biggl(\sqrt{\frac{N}{M}}\biggr).
\end{align*}
\section{Solving Sudoku}
We will apply the Grover's algorithm to solve a $2 \times 2$ binary Sudoku.
Considering all the possible ways of filling the board, the search space size
is $N = 2^{4}$, and a candidate solution $x = \ket{v_3, v_2,
  v_1, v_0}$ is a binary number in the interval $[0 \twodots N-1]$.
\begin{figure}[H]
  \centering
  \includegraphics[width=50pt]{Img/binary-sudoku.png}
  \caption{Binary Sudoku.}
\end{figure}
\subsection{Indicator function}
Let $W$ be the set of all valid
solutions (winning elements).
A candidate solution $x$ is considered valid ($x \in W$) if
the following conditions are met:
\begin{enumerate}
  \item
  no row may contain the same value twice ($v_0 \neq v_1, v_2 \neq v_3$);
  \item
  no columns may contain the same value twice ($v_0 \neq v_2, v_1 \neq v_3$).
\end{enumerate}
We can define our function $f$ as the indicator function of $W$:
\[
  f(x) = \bigg\{
  \begin{aligned}
    1, && \text{if} \; x \in W; \\
    0, && \text{otherwise}.
  \end{aligned}
\]

\subsection{Circuit code}
\begin{minted}[linenos]{Python}
l = 2    # Borad side.
n = l*l  # Borad size.

# Row clauses: 0 != 1, 2 != 3.
# Col clauses: 0 != 2, 1 != 3.
clause_list = [[0, 1], [0, 2], [1, 3], [2, 3]]

######## Function definitions. ########

# Use XOR to check every single clause.
def XOR(qc, q_in1, q_in2, q_out):

  qc.cx(q_in1, q_out)
  qc.cx(q_in2, q_out)

# Indicator function for binary Sudoku.
def f(qc):

# Check every single clause using XOR.
  c = 0
  for clause in clause_list:
    XOR(qc, clause[0], clause[1], clause_qubits[c])
    c += 1

  # Flip out qubit if all clauses are satified.
  qc.mct(clause_qubits, out_qubit)

# Uncompute clauses to reset clause-checking qbits to 0.
def uncompute(qc):

  # Repeat all the single clause check.
  c = 0
  for clause in clause_list:
    XOR(qc, clause[0], clause[1], clause_qubits[c])
    c += 1

# Apply f and reset clause-checkin qubits to 0.
def phase_oracle(qc):
  f(qc)
  uncompute(qc)

# Diffuser.
def diffuser(n):

  qc = QuantumCircuit(n)

  # Apply transformation |s> -> |00..0>.
  for qubit in range(n):
    qc.h(qubit)

  # Apply transformation |00..0> -> |11..1>.
  for qubit in range(n):
    qc.x(qubit)

  # Multi controlled Z-gate.
  qc.h(n-1)
  qc.mct(list(range(n-1)), n-1)
  qc.h(n-1)

  # Apply transformation |11..1> -> |00..0>.
  for qubit in range(n):
    qc.x(qubit)

  # Apply transformation |00..0> -> |s>.
  for qubit in range(n):
    qc.h(qubit)

  # Return the diffuser as a gate.
  U_s = qc.to_gate()
  U_s.name = "U_s"
  return U_s

######## Registers. ########

var_qubits = QuantumRegister(n, name='v')     # Variables.
clause_qubits = QuantumRegister(n, name='c')  # Clause-checks.
out_qubit = QuantumRegister(1, name='out')    # f(x).
cbits = ClassicalRegister(n, name='cbits')    # Classical bits.

######## Preparation. ########

# Create a quantum circuit.
qc = QuantumCircuit(var_qubits, clause_qubits, out_qubit, cbits)

# Superposition.
for q in var_qubits:
  qc.h(q)

# Initialize 'out' in state minus.
qc.x(out_qubit)
qc.h(out_qubit)
qc.barrier()

######## Amplification. ########

t = 2

for i in range(t):
  phase_oracle(qc)
  qc.append(diffuser(n), range(n))
  qc.barrier()

######## Measure. ########

# Measure the variable qubits.
qc.measure(var_qubits, cbits)

\end{minted}
\begin{figure}[H]
  \centering
  \includegraphics[width=345pt]{Img/circuit.png}
  \caption{Sudoku solving circuit.}
\end{figure}

\section{Simulation}
\begin{figure}[H]
  \centering
  \includegraphics[width=250pt]{Img/histogram.png}
  \caption{circuit simulation $t=2$.}
\end{figure}
Form the results of $1024$ simulations of our circuit we can observe that the
phase amplification trick it is working properly and it is really powerful.
In our case, following the theory and knowing in advance that $N = 2^4, M=2$,
we can find that:
\[
  t = \frac{\pi}{4} \sqrt{\frac{N}{M}}- \frac{1}{2} = \frac{\pi}{4}
  \sqrt{8}- \frac{1}{2} = 1.72 \approx 2.
\]
What happen with $t= 1$ or $t=3$? Results quality reduces:
\begin{figure}[H]
  \centering
  \begin{minipage}{.5\textwidth}
    \centering
    \includegraphics[width=6cm]{Img/histogram-t1.png}
    \caption{circuit simulation $t=1$.}
  \end{minipage}%
  \begin{minipage}{.5\textwidth}
    \centering
    \includegraphics[width=6cm]{Img/histogram-t3.png}
    \caption{circuit simulation $t=3$.}
  \end{minipage}
\end{figure}

\section{Error correction}
and comment the results obtained by running the circuit on the simulator (and
possibly on the hardware), comparing the effects of different sources of noise
(depolarizing errors on 1 or 2-qubit gates, relaxation, measurement). Try to
apply
some mitigation techniques to increase the quality of the final results.

\section{Conclusion}
In general, given a Sudoku, a rough upper bound for the search
space size is $n^{n^2},$ corresponding the number of all the possible ways
of filling an $n \times n$ board using $n$ values.
Therefore, a rough upper bound for a ``classical" board with $n=9$ is $9^{81}$,
that is near to the number of atoms in the universe;
exploiting the game rules is therefore essential to reduce the search space
and find a solution in a timely manner. This is the key idea behind all the
different and well-studied search techniques known in our days.

\end{document}
